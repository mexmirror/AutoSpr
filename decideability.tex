\section*{Entscheidbarkeit}
Eine Sprache ist \textbf{Turing-erkennbar} wenn es eine TM gibt, welche anhält und nur die Wörter der Sprache akzeptiert. Wörter welche nicht in der Sprache sind, werden von der TM entweder verworfen oder die \textit{TM hält nie an}. \newline
Eine Sprache ist \textbf{Turing-entscheidbar} wenn es eine TM gibt, welche Wörter in der Sprache akzeptiert und Wörter welche nicht in der Sprache sind, verwirft.
\paragraph{Entscheider} Ein Entscheider ist eine Turingmaschine, die auf jedem Input \(w \in \Sigma^*\) anhält. Eine Sprache heisst entscheidbar, wenn eine Entscheider sie erkennt. \newline
Eine nicht deterministische Turingmaschine ist ein Entscheider, wenn jede mögliche Berechnungsgeschichte terminiert. Eine Sprache ist enscheidbar, wenn sie von einer nicht deterministischen Turingmaschine entschieden wird.
\paragraph{Verifizierer} Ein Verifizierer für die Sprache \(A\) ist eine Turingmaschine \(V\) mit 
\begin{align*}
    A &= \{w \mid \exists c \in C^* \;\text{so dass}\; V\langle w,c \rangle \\ 
    &\text{akzeptiert}\}
\end{align*}
wobei \(C\) eine endliche Menge ist. Ein Verifizierer heisst polynomiell, wenn seine Laufzeit polynomiell ist in der Länge des Wortes \(w\). \textit{Gibt es einen polynomiellen Verifizierer, so ist die Sprache in NP.
}\\
Falls ein Verifizierer existiert, der das Problem in polynomieller Zeit verifizieren kann, kann das Problem auf einer nicht deterministischen Turingmaschine in polynomieller Zeit gelöst werden.
\paragraph{Akzeptanzproblem} Gesucht ist ein Automat, der entscheidet ob ein anderer Automat ein Input akzeptiert.
\begin{align*}
    A_{\text{DEA}} &= \{ \langle B, w \rangle \mid \\ 
    &\text{\textit{B} ist ein DEA und akzeptiert \textit{w}} \} \\
    A_{\text{NEA}} &= \{ \langle B, w \rangle \mid \\ 
    &\text{\textit{B} ist ein NEA und akzeptiert \textit{w}} \} \\
    A_{\text{REX}} &= \{ \langle R, w \rangle \mid \\
    &\text{\textit{R} ist ein REX und akzeptiert \textit{w}} \} \\
    A_{\text{CFG}} &= \{ \langle G, w \rangle \mid \text{die CFG \textit{G} erzeugt \textit{w}}\} \\
    A_{\text{TM}} &= \{ \langle M, w \rangle \mid \\
    &\text{\textit{M} ist eine TM und erkennt \textit{w}} \}
\end{align*}
\textit{DEA, NEA, REX} (Regular Expression) und \textit{CFG} sind entscheidbar, \textit{TM} nicht.
\paragraph{Gleichheitsproblem} Gesucht ist ein Automat der entscheidet, ob zwei andere Automaten die gleiche Sprache akzeptieren.
\begin{align*}
    EQ_\text{DEA} &= \{ \langle A, B \rangle \mid \text{\textit{A}, \textit{B} sind DEAs und} \\
    & L(A) = L(B) \} \\
    EQ_\text{CFG} &= \{ \langle G, H \rangle \mid \text{\textit{G}, \textit{H} sind CFGs und}\\
    & L(G) = L(H) \} \\
    EQ_\text{TM} &= \{ \langle M_1, M_2 \rangle \mid M_i \;\text{sind TMs und}\; \\
    &L(M_1) = L(M_2) \}
\end{align*}
\textit{DEA} ist entscheidbar, \textit{CFG, TM} nicht.
\paragraph{Leerheitsproblem} Gesucht ist ein Automat, der entscheidet, ob ein anderen Automat irgendein Wort akzeptiert.
\begin{align*}
    E_{\text{DEA}} &= \{ \langle A \rangle \mid \\
    &\text{\textit{A} ist ein DEA und} \; L(A) = \emptyset \} \\
    E_\text{CFG} &= \{  \langle G \rangle \mid \\
    & \text{\textit{G} ist eine CFG und} \; L(G) = \emptyset \} \\
    E_\text{TM} &= \{ \langle M \rangle \mid \\
    &\text{\textit{M} ist eine TM und} \; L(M) = \emptyset \}
\end{align*}
\textit{DEA, CFG} sind entscheidbar, \textit{TM} nicht.
\paragraph{Halteproblem} Das Halteproblem ist nicht entscheidbar
\begin{align*}
    \text{HALT}_{\text{TM}} &= \{ \langle M \rangle \mid \\
    &\text{\textit{M} ist eine TM und hält auf \textit{w}}\}
\end{align*}
\paragraph{Erzeugungsproblem} Gesucht ist ein Automat der entscheidet ob eine CFG alle Wörter erzeugt.
\begin{align*}
    ALL_{\text{CFG}} &= \{ \langle G \rangle \mid \text{\textit{G} ist CFG und} \\
    &L(G) = \Sigma^* \}
\end{align*}
\paragraph{Sprachprobleme}
\begin{itemize}
    \item Welche natürliche Zahlen sind Quadrate einer natürlichen Zahl \\
    \begin{align*}
        L\; & \text{eine Sprache über} \; \Sigma = \{0,1\} \\
        L &= \left\{ w \in \Sigma^* \left|\; 
        \begin{minipage}{0.15\textwidth}
            $w$ ist die Bin"ardarstellung einer Quadratzahl
        \end{minipage} \right. \right\} 
    \end{align*}
    \item Falls $n \in \mathbb{N}$ eine Quadratzahl ist, finde man die Wurzel
    \begin{align*}
                L\; & \text{eine Sprache über} \; \Sigma = \{0,1,:\} \\
        L &= \left\{ w \in \Sigma^* \left|\; 
        \begin{minipage}{0.15\textwidth}
            $w$ ist von der Form $w_1\texttt{:}w_2$, wobei $w_i$ Binardarstellungen von Zahlen $n_i$ sind mit $n_1=n_2^2$.
        \end{minipage} \right. \right\} 
    \end{align*}
    \item Hat die quadratische Darstellung $ax^2 +bx + c = 0$ mit $a,b,c \in \mathbb{N}$ reele Lösungen
        \begin{align*}
                L\; & \text{eine Sprache über} \; \Sigma = \{0,1,:\} \\
        L &= \left\{ w \in \Sigma^* \left|\; 
        \begin{minipage}{0.15\textwidth}
        $w$ ist von der Form $a\texttt{:}b\texttt{:}c$, wobei $a$, $b$ und $c$ Bin"ardarstellungen der Koeffizienten einer quadratischen Gleichung sind, die reelle L"osungen hat.
        \end{minipage} \right. \right\} 
    \end{align*} \\
    Diese Sprache kann mit einer TM entschieden werden, die die Diskriminante $b^2-4ac$ berechnet und im Zustand $q_{\text{accept}}$ stehen bleibt genau dann, wenn die Diskriminante $\ge 0$ ist.
    \item Hat die Gleichung $a^n + b^n = c^n$ ganzzahlige Lösungen, wobei mindestens eine der Zahlen $a,b,c$ grösser als 1 sein muss.
            \begin{align*}
                L\; & \text{eine Sprache über} \; \Sigma = \{0,1\} \\
        L &= \left\{ w \in \Sigma^* \left|\; 
        \begin{minipage}{0.15\textwidth}
       $w$ ist die Bin"ardarstellung einer nat"urlichen Zahl $n$, f"ur die die Gleichung $a^n+b^n=c^n$ eine ganzzahlige L"osung hat, wobei mindestens eine der Zahlen $a$, $b$ oder $c$ gr"osser als $1$ sein muss.
        \end{minipage} \right. \right\} 
    \end{align*} 
    \item Man finde die Primfaktoren einer Zahl n
                \begin{align*}
                L\; & \text{eine Sprache über} \; \Sigma = \{0,1,:\} \\
        L &= \left\{ w \in \Sigma^* \left|\; 
        \begin{minipage}{0.15\textwidth}
            $w$ ist von der Form $n\texttt{:}p_1\texttt{:}n_1\texttt{:}\dots\texttt{:}p_k\texttt{:}n_k$, und es gilt $n=p_1^{n_1}p_2^{n_2}\dots p_k^{n_k}$, wenn man die $p_i$ und $n_i$ als Bin"arzahlen interpretiert.
        \end{minipage} \right. \right\} 
    \end{align*}
    \item Goldbach-Vermutung: Jede Zahl > 2 könne als Summe von zwei Primzahlen geschrieben werden
    \begin{align*}
        L\; & \text{eine Sprache über} \; \Sigma = \{0,1\} \\
        L &= \left\{ w \in \Sigma^* \left|\; 
        \begin{minipage}{0.15\textwidth}
        $w$ ist die Bin"arcodierung einer einer geraden Zahl $n$, die als Summe von zwei Primzahlen geschrieben werden kann.
        \end{minipage} \right. \right\}         
    \end{align*}
    \item Primzahlenzwillings-Vermutung: Es gibt unendlich viele Primzahlenzwillinge
    \begin{align*}
                L\; & \text{eine Sprache über} \; \Sigma = \{0,1\} \\
        L &= \left\{ w \in \Sigma^* \left|\; 
        \begin{minipage}{0.15\textwidth}
            $w$ ist die Bin"arcodierung einer Zahl $n$, die eine Primzahl ist so, dass $n-2$ oder $n+2$ ebenfalls eine Primzahl ist.
        \end{minipage} \right. \right\}          
    \end{align*}
\end{itemize}