\section*{Turing Maschine} 
Formale Definition einer TM
\begin{align*}
    M = (Q, \Sigma, \Gamma, \delta, q_0, q_{accept}, q_{reject})
\end{align*}
\begin{itemize}
    \item \(Q\): Endliche Menge von Zuständen
    \item \(\Sigma\): Endliche Menge des Input-Alphabets (ohne \textvisiblespace)
    \item \(\Gamma\): Endliche Menge des Bandalphabets (mit \textvisiblespace), \(\Sigma \subset \Gamma\)
    \item \(\delta\): \(Q \times \Gamma \rightarrow Q \times \Gamma \times \{L, R\}\) als Übergangsfunktion
\end{itemize}
Die Turing Maschine verwendet ein oder mehrere unendlich Lange Bänder um Sprachen erkennen zu können.
\begin{tikzpicture}[->,>=stealth',shorten >=1pt,auto,node distance=3cm,semithick]
  \tikzstyle{every state}=[fill=none,draw=black,text=black]

  \node[state]                   (A)              {$q_0$};
  \node[state]                   (B) [right of=A] {$q$};

  \path (A)  edge  node {$a \rightarrow b,d$} (B);
\end{tikzpicture}\\
Von Zustand \(q_0\) wird in Zustand \(q\) gewechselt, wenn unter dem Kopf ein \(a\) gelesen wird. Dabei wird das \(a\) mit einem \(b\) überschrieben und der Kopf in Richtung \(d\) bewegt. 
\section*{Berechenbarkeit}
Sei \(\Sigma\) ein festes Alphabet, dann sind folgende Mengen abzählbar unendlich:
\begin{enumerate}
    \item Die Menge aller deterministischen endlichen Automaten.
    \item Die Menge aller nichtdeterministischen endlichen Automaten.
    \item Die Menge der regulären Sprachen.
    \item Die Menge aller kontextfreien Grammatiken.
    \item Die Menge aller kontextfreien Sprachen.
    \item Die Menge aller Stackautomaten.
    \item Die Menge aller Turingmaschinen.
\end{enumerate}
