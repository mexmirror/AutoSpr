\section*{Turing-Vollständigkeit}
LOOP in RISC-LOOP: Allgemein "ubersetzen wir $x_j:=x_i\pm c$ durch
\begin{itemize}
\item $i$ {\tt INCR} Befehle
\item Zugriff $r:=x_i$
\item $i$ {\tt DECR} Befehle
\item Rechnung $r:=r\pm c$
\item $j$ {\tt INCR} Befehle

\item Speicherung $x_j:=r$
\item $j$ {\tt DECR} Befehle
\end{itemize}
Analog "ubersetzen wir $\text{\tt LOOP} x_i$ durch
\begin{itemize}
\item $i$ {\tt INCR} Befehle
\item Zugriff $r:=x_i$
\item $i$ {\tt DECR} Befehle
\item Schleifenbefehl $\text{\tt LOOP }r$.
\end{itemize}
Auf diese Weise l"asst sich jedes LOOP-Programm in eine RISC-LOOP Programm
"ubersetzen, RISC-LOOP ist also mindestens so leistungsf"ahig wie LOOP. (LOOP ist jedoch nicht Turing-Vollständig)
